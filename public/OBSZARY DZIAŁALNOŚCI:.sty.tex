OBSZARY DZIAŁALNOŚCI: 
1.	PRAWO MEDYCZNE – BŁĘDY MEDYCZNE (diagnostyczne, terapeutyczne, zabiegowe i organizacyjne)
DLA PACJENTA 
- analiza dokumentacji medycznej i ocena zasadności roszczeń
- dochodzenie roszczeń cywilnych o odszkodowania i zadośćuczynienia 
- prowadzenie postępowań w przypadku naruszenia praw pacjenta
- mediacje i negocjacje z podmiotami leczniczymi i ubezpieczycielami
- postępowanie dyscyplinarne przed organami samorządu zawodowego 
- postępowania przed Rzecznikiem Praw Pacjenta oraz Funduszem Kompensacyjnym Zdarzeń Medycznych
- postępowanie karne medyczne: 
--reprezentacja pokrzywdzonych pacjentów
-- doradztwo w sprawach dotyczących tajemnicy lekarskiej, fałszywej dokumentacji medycznej czy zgody pacjenta

DLA PODMIOTÓW MEDYCZNYCH 
- doradztwo prawne w zakresie odpowiedzialności zawodowej i ryzyk ubezpieczeniowych 
- reprezentacja w sporach cywilnych 
- - postępowanie karne medyczne: 
--obrona lekarzy i personelu medycznego w sprawach o nieumyślne spowodowanie śmierci (art. 155 k.k.), narażenie pacjenta na niebezpieczeństwo, (art. 160 k.k.), nienależyte wykonanie obowiązków zawodowych (art. 231 k.k.),
-- doradztwo w sprawach dotyczących tajemnicy lekarskiej, fałszywej dokumentacji medycznej czy zgody pacjenta
- przygotowywanie i opiniowanie umów medycznych
- szkolenia z zakresu prawa medycznego, obowiązku informacyjnego dla pacjentów
- opracowywanie procedur dotyczących reklamacji, zgód pacjentów, prowadzenia dokumentacji medycznej
- obsługa podmiotów leczniczych 
2. PRAWO ODSZKODOWAWCZE :
- wypadki komunikacyjne: odszkodowanie i zadośćuczynienie dla poszkodowanych, roszczenia przeciwko sprawcom oraz zakładom ubezpieczeń (AC, OC NNW), reprezentacja w postępowaniach sądowych
- wypadki przy pracy i rolnictwie: roszczenia z tytułu wypadków przy pracy, chorób zawodowych, odszkodowania ZUS i ubezpieczeń prywatnych, reprezentacja przed ZUS i sądem
- odpowiedzialność cywilna osób prywatnych i przedsiębiorstw: szkody wyrządzone przez osoby trzecie, pracowników, zwierzęta, rzeczy, odpowiedzialność za szkody wyrządzone przez przedsiębiorców i instytucje, analiza umów i polic ubezpieczeniowych
- wypadki w miejscach publicznych: poślizgnięcie, potknięcie, upadki – roszczenia wobec zarządców dróg, sklepów, urzędów, dochodzenia odszkodowań za szkody na osobie i mieniu 
- odszkodowania majątkowe i niemajątkowe: zadośćuczynienie za doznaną krzywdę i cierpienie fizyczne i psychiczne, odszkodowania za utracone dochody, koszty leczenia, rehabilitacji, opieki, renty z tytułu utraty zdolności do pracy lub zwiększonych potrzeb życiowych
- spory z ubezpieczycielami: odwołania od decyzji o odmowie wypłaty lub zaniżeniu, reprezentacja w negocjacjach i postępowaniu sądowym 
3.WINDYKACJA NALEŻNOŚCI:
- windykacja przedsądowa: analiza zasadności roszczenia i dokumentacji, sporządzenie wezwań do zapłaty, negocjacje, ustalenie ugód ratalnych, monitorowanie terminów płatności i przypomnienie dłużnikom o zobowiązaniach
- windykacja sądowa: pozwy o zapłatę, odpowiedzi na pozew, reprezentacja przed sądami, prowadzenie spraw w postępowaniu nakazowym, upominawczym i uproszczonym, uzyskiwanie tytułów egzekucyjnych (wyrok, nakaz zapłaty, ugoda)
- windykacja komornicza: kierowanie spraw do komornika i nadzór nad postępowaniem egzekucyjnym, wnioski egzekucyjne i skargi na czynności komornicze, współpraca z zaufanymi komornikami
4. PRAWO RODZINNE, RZECZOWE I SPADKI:
Rodzinne:
 - sprawy rozwodowe i separacyjne: prowadzenie spraw o rozwód z orzeczeniem o winie lub bez orzekania o winie, sprawy o separacje, ustalenie sposobu korzystania ze wspólnego mieszkania po rozwodzie, 
- podział majątku wspólnego: negocjacje i reprezentacja w sprawach o podział majątku po rozwodzie lub separacji,  ustalenie wartości majątku i udziałów małżonków, rozliczenia nakładów i długów małżeńskich, 
- władza rodzicielska i kontakty z dziećmi: sprawy o powierzenie władzy rodzicielskiej i ustalenie miejsca zamieszkania dziecka, regulowanie lub ograniczenie kontaktów z dzieckiem, reprezentacja w sporach między rodzicami,
- alimenty: dochodzenie alimentów na rzecz dzieci, małżonków lub rodziców, podwyższenie, obniżenie lub uchylenie obowiązku alimentacyjnego, egzekucja zaległych alimentów,
- ustalenie i zaprzeczenie ojcostwa/macierzyństwa: prowadzenie postępowań o ustalenie ojcostwa lub zaprzeczenie ojcostwa, ustalenie praw i obowiązków wobec dziecka, regulacja sytuacji prawnej dziecka w aktach stanu cywilnego
- przysposobienie (adopcja): kompleksowa obsługa spraw adopcyjnych, przygotowanie dokumentacji i reprezentacja w sądzie rodzinnym, doradztwo w zakresie prawnych skutków przysposobienia 
- przemoc w rodzinie i ochrona prawna: wnioski o zakaz zbliżania się, eksmisję sprawcy przemocy, wsparcie w uzyskaniu natychmiastowej ochrony sądowej, współpraca z organami ściągania i środkami pomocy
Rzeczowe:
- własność nieruchomości i ruchomości: zniesienie współwłasności, zasiedzenie, ochrona przed naruszeniem własności i posiadania, 
- służebności i ograniczone prawa rzeczowe: ustanowienie i znoszenie służebności gruntowych i osobistych, droga konieczna i przesyłu, analiza umów ustanawiających służebność 
- posiadanie i ochrona posiadania: roszczenia negatoryjne i windykacyjne, przywrócenie naruszonego posiadania, reprezentacja w postępowaniach sądowych
- regulacja stanu prawnego nieruchomości: postępowanie wieczystoksięgowe, ustalenie właściciela nieruchomości o nieustalonym stanie prawnym, 
- umowy dotyczące nieruchomości: przygotowywanie umów sprzedaży, darowizny, zmiany, dożywocia, doradztwo w zakresie kupna i sprzedaży nieruchomości, reprezentacja przed sądem
Spadki:
- stwierdzenie nabycia spadku i dział spadku: prowadzenie postępowań o stwierdzenie nabycia spadku po zmarłym, kompleksowa obsługa spraw o dział spadku między spadkobiercami, ustalenie udziałów, rozliczenie nakładów i spłat, reprezentacja w postępowaniach sądowych;
- odrzucenie i przyjęcie spadku: doradztwo w zakresie wyboru najkorzystniejszej formy przyjęcia spadku, sporządzenie oświadczeń spadkowych, reprezentacja w postępowaniach sądowych;
- zachowek: dochodzenie roszczeń o zachowek dla uprawnionego, obrona przed nieuzasadnionymi roszczeniami o zachowek, negocjacje i ugody w sprawach rodzinnych;
- nieważność testamentu;
- odrzucenie spadku przez małoletniego
5. PRAWO KARNE:
- obrona podejrzanych (postępowanie przygotowawcze), obrona oskarżonych (postępowanie sądowe), reprezentacja, sporządzanie zażaleń, apelacji, kasacji
- reprezentacja osób pokrzywdzonych: występowanie w charakterze pełnomocnika pokrzywdzonych, pomoc w przygotowaniu zawiadomień o przestępstwie
- postępowanie wykonawcze (po wyroku): wnioski o odroczenie lub przerwę w wykonaniu kary, warunkowe przedterminowe zwolnienie, zmiana kary pozbawienia wolności na karę ograniczenia wolności, wniosek o odbywanie kry w systemie dozoru elektronicznego (SDE), zatarcie skazania i usuwanie wpisów z KRK
- prawo karne nieletnich: obrona nieletnich przed sądami rodzinnymi, wnioski o środki wychowawcze zamiast poprawczych 
6. MEDIACJE 
- rodzinne: spory małżeńskie, kwestie opieki nad dziećmi, kontakty i alimenty, wsparcie w osiągnięciu porozumienia przy rozwodzie lub separacji;
- cywilne i majątkowe: spory o własność majątku, roszczenia finansowe, konflikty między współwłaścicielami, sąsiadami, ustalenie warunków spłaty, ugody
- gospodarcze: spory między przedsiębiorcami, negocjowanie ugód,
- pracownicze: spory w kwestiach zatrudnienia, mobbingu, wypowiedzeń











O MNIE:
Adwokat Anna Romańska - Krysa: absolwentka Wydziału Prawa i Administracji Uniwersytetu Śląskiego w Katowicach i studiów podyplomowych – Prawo w ochronie zdrowia na powyższym wydziale. Swoją wiedzę i doświadczenie wzbogacała na licznych kursach i szkoleniach z zakresu negocjacji i mediacji. Od 2022 roku wpisana na listę stałych mediatorów sądowych Prezesa Sądu Okręgowego w Katowicach i Warszawie. W 2025 roku otrzymała wpis na listę adwokatów Okręgowej Rady Adwokackiej w Katowicach, pod numerem….
Specjalizuje się w sprawach z zakresu prawa medycznego. Pomaga zarówno podmiotom medycznym, jak również poszkodowanym pacjentom. Od lat współpracuje z kancelariami zajmującymi się błędami medycznymi, a jej prace naukowe, dotyczące tej tematyki, były publikowane w czasopismach medycznych.
Adwokat Romańska-Krysa posiada wieloletnie doświadczenie w sprawach z zakresu postępowań sądowych, zarówno cywilnych, jak i karnych. Specjalizuje się w prawie cywilnym - zwłaszcza w dochodzeniu roszczeń odszkodowawczych, windykacji roszczeń i w sprawach z zakresu prawa rodzinnego. 
Do każdej sprawy podchodzi z wielkim zaangażowaniem i determinacją, a jej indywidualne podejście do każdego klienta zawsze pozwala wypracować jak najlepszą strategię. 
Biegle posługuję się językiem angielskim.
